
\section{Éléments P1-Lagrange}
\noindent Dans cette section nous réécrirons les problèmes variationnels obtenus précédemment en P1-Lagrange avec une méthode de Galerkin. Nous supposerons la méthode des éléments finis P1 en 2D déjà connue est seuls les calculs les plus importants seront détaillés.\\

\noindent Nous rappelons l'élément de référence $2D$ utilisé dans la suite des développements :
\begin{figure}[H]
	\centering
	\incfig{0.5}{Images/triangle_ref}
\caption[Élément de référence en \textbf{2D}]{Élément de référence en \textbf{2D}}
\label{fig:elementreference}
\end{figure}
\noindent Où la fonction de passage entre l'élément de référence $\widehat{K}$ et un élément $K$ est
\begin{equation}
	\mathcal{T}_{\widehat{K}\mapsto K} (\xi, \eta) := \myvector{x}{y}{} =  \left[\begin{array}{cc}
	x_2-x_1 & x_3-x_1\\
	y_2-y_1 & y_3-y_1\\
	\end{array}\right] \myvector{\xi}{\eta}{} + \myvector{x_1}{y_1}{}
\end{equation}
avec comme matrice jacobienne
\begin{equation}
	\mathbf{J}_{\mathcal{T}_{\widehat{K}\mapsto K}} := \text{Jac}\left[\mathcal{T}_{\widehat{K}\mapsto K}\right] = \left[\begin{array}{cc}
	x_2-x_1 & x_3-x_1\\
	y_2-y_1 & y_3-y_1\\
	\end{array}\right].
\end{equation}
Remarquons que 
\begin{equation}
	\text{det}\left(\mathbf{J}_{\mathcal{T}_{\widehat{K}\mapsto K}}\right) = 2\aire{K} =: 2\,\text{Aire} (K)
\end{equation}
et donc nous avons
\begin{equation*}
	\mathbf{J}_{\mathcal{T}_{\widehat{K}\mapsto K}}^{-1}=\frac{1}{2\aire{K}}\text{Com}\left(\mathbf{J}_{\mathcal{T}_{\widehat{K}\mapsto K}}\right)^{T}=\frac{1}{2\aire{K}}\left[\begin{array}{cc}
	y_3-y_1 & x_1-x_3\\
	y_1-y_2 & x_2-x_1\\
\end{array}\right].
\end{equation*}
Nous rappelons aussi les 3 fonctions d'interpolations P1 sur l'élément de référence, ainsi que leur gradients
\begin{center}
\begingroup
\setlength{\tabcolsep}{10pt} % Default value: 6pt
\renewcommand{\arraystretch}{1.5} % Default value: 1
\begin{tabular}{|c|c|c|}
	\hline
	$i$ & $\widehat{\varphi_i}$ & $\widehat{\nabla} \widehat{\varphi_i}$\\
	\hline
	$1$ & $1-\xi-\eta$ & $\left[-1, -1\right]^{T}$\\
	$2$ & $\xi$ & $\left[1, 0\right]^{T}$\\
	$3$ & $\eta$ & $\left[0, 1\right]^{T}$\\
	\hline	
\end{tabular}
\endgroup
\end{center}

\noindent Remarquons aussi que $\forall i, j \in \left\{1, 2, 3\right\}$
\begin{equation}
\int_{\widehat{K}}\widehat{\varphi_i}\widehat{\varphi_j} = \delta_{ij}\frac{1}{24} + \frac{1}{24}\label{eq:integrale_phiij}
\end{equation}
Nous rappelons aussi la formule de changement de variables dans une intégrale en annexe \eqref{op:changementdevariables}.

\subsection{Système dans $\Omega_w$}
\noindent Nous définissons la discrétisation de $\zeta$ sur un élément $K$ comme 
\begin{equation}
	\left[\zeta\right]^{K} := \left\{\zeta_j\right\}_{j=\overline{1, 3}}
\end{equation}
et une approximation de la fonction $\chi_{dam}$ sur $K$ comme
\begin{equation}
	\widetilde{\chi_{dam}}^{K} := \sum_{j=1}^{3}\frac{\chi_{dam} (x_j, y_j)}{3}.
\end{equation}

\subsubsection{Partie temporelle}
\noindent La partie temporelle de la formulation variationnelle est contenue dans $\bld{a^{time/dam}}$ \eqref{eq:a_time_dam_1}. Nous notons l'opérateur de dérivée seconde $T^{(2)}$ et l'opérateur de dérivée première $T^{(1)}$ : nous y reviendrons ensuite.\\
Après calculs, $\bld{a^{time/dam}}$ en formulation variationnelle élémentaire devient
\begin{equation}
	\left[\bld{a^{time/dam}}\right]^{K} = \aire{K}R T^{(2)} \left[\zeta\right]^{K} + \widetilde{\chi_{dam}}^{K}\aire{K} R T^{(1)} \left[\zeta\right]^{K}
\end{equation}
avec
\begin{equation}
R = \frac{1}{24}\begin{bmatrix}
1 & 2 & 2\\
2 & 1 & 2\\
2 & 2 & 1
\end{bmatrix}
\end{equation}
\subsubsection{Partie spatiale}
\noindent Prenons en premier lieu la partie sur $\widetilde{\Omega_w}$ de \eqref{eq:a_space_1} :
\begin{equation}
	\left[W\right]_{ij} = \iint_{K} \nabla \varphi_i\cdot\nabla \varphi_j= 2\aire{K} \iint_{\widehat{K}}  \mathbf{J}_{\mathcal{T}_{\widehat{K}\mapsto K}}^{-T}\widehat{\nabla} \widehat{\varphi_i} \cdot \mathbf{J}_{\mathcal{T}_{\widehat{K}\mapsto K}}^{-T} \widehat{\nabla} \widehat{\varphi_j}
\end{equation}

\noindent Prenons en premier lieu \eqref{eq:a_sb_1} :
\begin{equation}
\left[R\right]_{ij} = \iint_{K} \nabla \varphi_i\cdot\nabla \varphi_j \,dxdy = 2\aire{K} \iint_{\widehat{K}}  \left(\nabla \varphi_i\right)^{T}\mathbf{J}_{\mathcal{T}_{\widehat{K}\mapsto K}}^{-1} \mathbf{J}_{\mathcal{T}_{\widehat{K}\mapsto K}}^{-T} \nabla \varphi_j  \,d\xi d\eta
\end{equation}

\subsection{Système dans $\Omega_b$}



\subsection{Schéma Newmark}

\begin{equation}
	M \ddot{u} + C \dot{u} + Au = F
\end{equation}

\begin{subnumcases}{}
	\dot{u}_{n+1} = \dot{u}_n + \Delta t \ddot{u}_{\alpha}\\
	\text{avec }\ddot{u}_{\alpha} = \alpha\ddot{u}_{n} + \left(1- \alpha\right)\ddot{u}_{n+1}\\
	u_{n+1} = u_n + \Delta t \dot{u}_{n} + \frac{1}{2}\Delta t ^2 \ddot{u}_{\beta}\\
	\text{avec }\ddot{u}_{\beta} = 2 \beta\ddot{u}_{n} + \left(1- 2\beta\right)\ddot{u}_{n+1}
\end{subnumcases}

\begin{equation}
	u_{n+1} = u_n + \Delta t \partial_t u_n + \frac{\Delta t^2}{2} \partial_{2t} u_n + \frac{\Delta t^3}{6} \partial_{3t} u_n + \frac{\Delta t^4}{24} \partial_{4t} u_n + \frac{\Delta t^5}{120} \partial_{5t} u_n
\end{equation}

\begin{equation}
u_{n+1} = u_{n-1} + 2\Delta t \partial_t u_{n-1} + 2\Delta t^2 \partial_{2t} u_{n-1} + \frac{4\Delta t^3}{3} \partial_{3t} u_{n-1} + \frac{2\Delta t^4}{3} \partial_{4t} u_{n-1} + \frac{4\Delta t^5}{15} \partial_{5t} u_{n-1}
\end{equation}

\begin{equation}
	\ddddot{u}
\end{equation}
