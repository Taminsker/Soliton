
\section{Solutions progressives et analytiques de l'équation des ondes}
\noindent Cette section est inspirée d'un document que m'a transmis Mr Ricchiuto intitulé "Travelling solutions : linear shallow water equations".\\

\noindent Le système formé de \eqref{eq:mass} et \eqref{eq:massflux} peut être réécrit de façon simple dans $\Omega_w$ (sans flotteur) dans une direction $\boldvec{d} = \left[\mathbf{d_1}, \mathbf{d_2}\right]^{T}$ :
\begin{subnumcases}{}
	\partial_t\zeta + \mathbf{d_1}\partial_x q_1 + \mathbf{d_2}\partial_y q_2 = 0, \label{eq:water_1}\\
	\partial_t q_1 + \mathbf{d_1}c_0^2\partial_x\zeta = 0,\label{eq:water_2}\\
	\partial_t q_2 + \mathbf{d_2}c_0^2\partial_y\zeta = 0,\label{eq:water_3}
\end{subnumcases}
avec $c_0^2 = gd_0 = gh_0$. La mise sous forme d'un système est relativement simple
\begin{equation}
	\partial_t W + \mathbf{d_1}A\partial_x W + \mathbf{d_2}B\partial_y W = \bld{0}, \hspace{6mm}W = \myvector{\zeta}{q_1}{q_2}, \hspace{6mm}A = \begin{bmatrix}
	0 & 1 & 0\\
	c_0^2 & 0 & 0\\
	0 & 0 & 0
	\end{bmatrix}_{\mathbb{R}^{3\times 3}}, \text{ et }\hspace{6mm} B = \begin{bmatrix}
	0 & 0 & 1\\
	0 & 0 & 0\\
	c_0^2 & 0 & 0
	\end{bmatrix}_{\mathbb{R}^{3\times 3}}\label{eq:sys_1}
\end{equation}

\noindent Les matrices $A$ et $B$ peuvent être écrites comme suit
\begin{equation}
A = L_{A} D_A R_A, \hspace{3mm} L_A = \begin{bmatrix}
\frac{1}{c_0} & -\frac{1}{c_0}\\
1 & 1\\
0 & 0
\end{bmatrix}_{\mathbb{R}^{3\times 2}}, \hspace{3mm}D_A= \begin{bmatrix}
c_0 & 0\\
0 & -c_0
\end{bmatrix}_{\mathbb{R}^{2\times 2}}, \hspace{3mm}R_A =\begin{bmatrix}
\frac{c_0}{2} & \frac{1}{2} & 0\\
-\frac{c_0}{2} & \frac{1}{2} & 0
\end{bmatrix}_{\mathbb{R}^{2\times 3}}
\end{equation}

\begin{equation}
B = L_B D_B R_B, \hspace{3mm} L_B = \begin{bmatrix}
\frac{1}{c_0} & -\frac{1}{c_0}\\
0 & 0\\
1 & 1
\end{bmatrix}_{\mathbb{R}^{3\times 2}}, \hspace{3mm}D_B = \begin{bmatrix}
c_0 & 0\\
0 & -c_0
\end{bmatrix}_{\mathbb{R}^{2\times 2}}, \hspace{3mm}R_B = \frac{1}{2}\begin{bmatrix}
\frac{c_0}{2} & 0 & \frac{1}{2}\\
-\frac{c_0}{2} & 0 & \frac{1}{2}
\end{bmatrix}_{\mathbb{R}^{2\times 3}}
\end{equation}
Introduisons maintenant les variables caractéristiques
\begin{equation}
N^{\bld{x}} = \myvector{\eta^{+\bld{x}}}{\eta^{-\bld{x}}}{}:= R_A W = \myvector{\frac{1}{2}\left(q_1 + c_0\zeta\right)}{\frac{1}{2}\left(q_1 - c_0\zeta\right)}{},
\end{equation}
\begin{equation}
N^{\bld{y}} =\myvector{\eta^{+\bld{y}}}{\eta^{-\bld{y}}}{} := R_B W = \myvector{\frac{1}{2}\left(q_2 + c_0\zeta\right)}{\frac{1}{2}\left(q_2 - c_0\zeta\right)}{}.
\end{equation}
Nous pouvons remarquer que
\begin{subnumcases}{}
\zeta = \frac{1}{2c_0}\left(\eta^{+\bld{x}} - \eta^{-\bld{x}} +\eta^{+\bld{y}} - \eta^{-\bld{y}}\right),\\
q_1 = \eta^{+\bld{x}} + \eta^{-\bld{x}},\\
q_2 = \eta^{+\bld{y}} + \eta^{-\bld{y}}.
\end{subnumcases}
\noindent Nous obtenons deux caractérisations de $W$ (une dans la direction $\boldvec{x}$ et une dans la direction $\boldvec{y}$)
\begin{equation}
W^{\bld{x}} = L_A N^{\bld{x}}\hspace{4mm}\text{et}\hspace{4mm}W^{\bld{y}} = L_B N^{\bld{y}}.
\end{equation}
Nous pouvons dès lors dégager deux systèmes, l'un en multipliant \eqref{eq:sys_1} par $R_A$ et en remplaçant $W$ par $W^{\bld{x}}$ et l'autre $R_B$ et $W^{\bld{y}}$
\begin{subnumcases}{}
\partial_tN^{\bld{x}} + \mathbf{d_1}\partial_x D_A N^{\bld{x}} + \mathbf{d_2}\partial_y R_A B L_A N^{\bld{x}}  = 0\\
\partial_tN^{\bld{y}} + \mathbf{d_1}\partial_x R_BA L_B N^{\bld{y}} + \mathbf{d_2}\partial_y D_B N^{\bld{y}}  = 0
\end{subnumcases}
Or nous avons $ R_AB L_A =  R_BA L_B =\bld{ 0}$. Nous obtenons donc les systèmes\\
\begin{minipage}{0.45\textwidth}
	\begin{subnumcases}{}
	\left(\partial_t  + \boldvec{v}^{\bld{d_1}}\cdot \nabla \right)\eta^{+\bld{x}} = 0\\
	\left(\partial_t  - \boldvec{v}^{\bld{d_1}}\cdot \nabla \right)\eta^{-\bld{x}} = 0
	\end{subnumcases}
\end{minipage}
\begin{minipage}{0.45\textwidth}
	\begin{subnumcases}{}	
	\left(\partial_t  + \boldvec{v}^{\bld{d_2}}\cdot \nabla \right)\eta^{+\bld{y}} = 0\\
	\left(\partial_t  - \boldvec{v}^{\bld{d_2}}\cdot \nabla \right)\eta^{-\bld{y}} = 0
	\end{subnumcases}
\end{minipage}\vspace*{3mm}\\
\noindent où $\boldvec{v}_{\bld{d_1}} = \left(c_0\mathbf{d_1}, 0\right)^{T}$ et $\boldvec{v}_{\bld{d_2}} = \left(0, c_0\mathbf{d_2}\right)^{T}$. Ainsi nous avons les solutions générales\\
\begin{minipage}{0.49\textwidth}
	\begin{subnumcases}{}
	\eta^{+\bld{x}}(\boldvec{x}, t) = \eta^{+\bld{x}}_0(\boldvec{x} - \boldvec{v}_{\bld{d_1}} t)\\
	\eta^{-\bld{x}}(\boldvec{x}, t) = \eta^{-\bld{x}}_0(\boldvec{x} + \boldvec{v}_{\bld{d_1}} t)
	\end{subnumcases}
\end{minipage}
\begin{minipage}{0.49\textwidth}
	\begin{subnumcases}{}
	\eta^{+\bld{y}}(\boldvec{x}, t) = \eta^{+\bld{y}}_0(\boldvec{x} - \boldvec{v}_{\bld{d_2}} t)\\
	\eta^{-\bld{y}}(\boldvec{x}, t) = \eta^{-\bld{y}}_0(\boldvec{x} + \boldvec{v}_{\bld{d_2}} t)
	\end{subnumcases}
\end{minipage}\vspace*{3mm}\\
Ainsi
\begin{subnumcases}{}
	\zeta(\boldvec{x}, t) = \frac{\eta^{+\bld{x}}_0(\boldvec{x} - \boldvec{v}_{\bld{d_1}} t)-\eta^{-\bld{x}}_0(\boldvec{x} + \boldvec{v}_{\bld{d_1}} t) + \eta^{+\bld{y}}_0(\boldvec{x} - \boldvec{v}_{\bld{d_2}} t)-\eta^{-\bld{y}}_0(\boldvec{x} + \boldvec{v}_{\bld{d_2}} t)}{2c_0}\\
	\boldvec{q}(\boldvec{x}, t) = \myvector{\eta^{+\bld{x}}_0(\boldvec{x} - \boldvec{v}_{\bld{d_1}} t)+\eta^{-\bld{x}}_0(\boldvec{x} + \boldvec{v}_{\bld{d_1}} t)}{\eta^{+\bld{y}}_0(\boldvec{x} - \boldvec{v}_{\bld{d_2}} t)+\eta^{-\bld{y}}_0(\boldvec{x} + \boldvec{v}_{\bld{d_2}} t)}{}
\end{subnumcases}
%\begin{subnumcases}{}
%	\zeta(x, y, t) = \frac{\eta^{+\bld{x}}_0(x - c_0t, y) - \eta^{-\bld{x}}_0(x + c_0t, y)}{2c_0} + \frac{\eta^{+\bld{y}}_0(x, y-c_0t) - \eta^{-\bld{y}}_0(x, y+c_0t)}{2c_0}\\
%	q_1(x, y, t) = \eta^{+\bld{x}}_0(x - c_0t, y) + \eta^{-\bld{x}}_0(x + c_0t, y)\\
%	q_2 (x, y, t) = \eta^{+\bld{y}}_0(x, y-c_0t) + \eta^{-\bld{y}}_0(x, y+c_0t)
%\end{subnumcases}

%\vspace*{5cm}
%Les matrices $A_0$ et $A_1$ peuvent être diagonalisées comme suit
%\begin{equation}
%	A_0 = P_0 D_0 P_0^{-1}, \hspace{3mm} P_0 = \frac{1}{c_0}\begin{bmatrix}
%	0 & 1 & -1\\
%	0 & c_0 & c_0\\
%	c_0 & 0 & 0
%	\end{bmatrix}, \hspace{3mm}D_0 = c_0\begin{bmatrix}
%	0 & 0 & 0\\
%	0 & 1 & 0\\
%	0 & 0 & -1
%	\end{bmatrix}, \hspace{3mm}P_0^{-1} = \frac{1}{2}\begin{bmatrix}
%	0 & 0 & 2\\
%	c_0 & 1 & 0\\
%	-c_0 & 1 & 0
%	\end{bmatrix}
%\end{equation}
%\begin{equation}
%A_1 = P_1 D_1 P_1^{-1}, \hspace{3mm} P_1 = \frac{1}{c_0}\begin{bmatrix}
%0 & 1 & -1\\
%c_0 & 0 & 0 \\
%0 & c_0 & c_0
%\end{bmatrix}, \hspace{3mm}D_1 = c_0\begin{bmatrix}
%0 & 0 & 0\\
%0 & 1 & 0\\
%0 & 0 & -1
%\end{bmatrix}, \hspace{3mm}P_1^{-1} = \frac{1}{2}\begin{bmatrix}
%0 & 2 & 0\\
%c_0 & 0 & 1\\
%-c_0 & 0 & 1
%\end{bmatrix}
%\end{equation}
%Introduisons maintenant les variables caractéristiques
%\begin{equation}
%	N^{\bld{x}} = \myvector{q_2}{\eta^{+\bld{x}}}{\eta^{-\bld{x}}}:= P_0^{-1} W = \myvector{q_2}{\frac{1}{2}\left(q_1 + c_0\zeta\right)}{\frac{1}{2}\left(q_1 - c_0\zeta\right)},
%\end{equation}
%\begin{equation}
%	N^{\bld{y}} =\myvector{q_1}{\eta^{+\bld{y}}}{\eta^{-\bld{y}}} := P_1^{-1} W = \myvector{q_1}{\frac{1}{2}\left(q_2 + c_0\zeta\right)}{\frac{1}{2}\left(q_2 - c_0\zeta\right)}.
%\end{equation}
%Nous pouvons remarquer que 
%\begin{subnumcases}{}
%	\zeta = \frac{1}{2c_0}\left(\eta^{+\bld{x}} - \eta^{-\bld{x}} +\eta^{+\bld{y}} - \eta^{-\bld{y}}\right),\\
%	q_1 = \eta^{-\bld{x}} + \eta^{+\bld{x}},\\
% 	q_2 = \eta^{-\bld{y}} + \eta^{+\bld{y}}.
%\end{subnumcases}\\
%
%\noindent Nous obtenons deux caractérisations de $W$ (une dans la direction $\boldvec{x}$ et une dans la direction $\boldvec{y}$)
%\begin{equation}
%	W^{\bld{x}} = P_0 N^{\bld{x}}\hspace{4mm}\text{et}\hspace{4mm}W^{\bld{y}} = P_1 N^{\bld{y}}.
%\end{equation}
%Nous pouvons dès lors dégager deux systèmes, l'un en multipliant \eqref{eq:sys_1} par $P_0^{-1}$ et en remplaçant $W$ par $W^{\bld{x}}$ et l'autre $P_1^{-1}$ et $W^{\bld{y}}$
%\begin{subnumcases}{}
%	N^{\bld{x}}_t + \partial_x D_0 N^{\bld{x}} + \partial_y P_0^{-1}A_1 P_0 N^{\bld{x}}  = 0\\
%	N^{\bld{y}}_t + \partial_x P_1^{-1}A_0 P_1 N^{\bld{y}} + \partial_y D_1 N^{\bld{y}}  = 0
%\end{subnumcases}
%et obtenir en négligeant les premières équations\\
%%\begin{minipage}{0.49\linewidth}
%\begin{subnumcases}{}
%%		\partial_t \left(\eta^{+\bld{y}} + \eta^{-\bld{y}}\right) + c_0 \partial_y \left(\eta^{+\bld{x}} - \eta^{-\bld{x}}\right) = 0\label{s:11}\\
%		\partial_t \eta^{+\bld{x}} + c_0 \partial_x \eta^{+\bld{x}}
%		=- \frac{c_0}{2} \partial_y \left(\eta^{+\bld{y}} + \eta^{-\bld{y}}\right) \label{s:12}\\
%		\partial_t \eta^{-\bld{x}} 
%		- c_0 \partial_x \eta^{-\bld{x}} =+
%		\frac{c_0}{2} \partial_y \left(\eta^{+\bld{y}} + \eta^{-\bld{y}}\right)	\label{s:13}\\[1em]
%%		\partial_t \left(\eta^{+\bld{x}} + \eta^{-\bld{x}}\right) + c_0 \partial_x \left(\eta^{+\bld{y}} - \eta^{-\bld{y}}\right) =0 \label{s:21}\\
%		\partial_t \eta^{+\bld{y}} 
%		+ c_0 \partial_y \eta^{+\bld{y}} = -\frac{c_0}{2} \partial_x \left(\eta^{+\bld{x}} + \eta^{-\bld{x}}\right)\label{s:22}\\
%		\partial_t \eta^{-\bld{y}} 
%		- c_0 \partial_y \eta^{-\bld{y}}= +\frac{c_0}{2} \partial_x \left(\eta^{+\bld{x}} + \eta^{-\bld{x}}\right) \label{s:23}
%\end{subnumcases}
%
%%\begin{subnumcases}{}
%%	\eqref{s:22}+\eqref{s:23}-\eqref{s:11} \Longrightarrow \partial_y \left(\eta^{+\bld{y}} - \eta^{-\bld{y}}\right) = \partial_y \left(\eta^{+\bld{x}} -\eta^{-\bld{x}}\right)\\
%%	\eqref{s:12}+\eqref{s:13}-\eqref{s:21} \Longrightarrow \partial_x \left(\eta^{+\bld{y}} - \eta^{-\bld{y}}\right) = \partial_x \left(\eta^{+\bld{x}} -\eta^{-\bld{x}}\right)
%%\end{subnumcases}
%%Ainsi
%%\begin{subnumcases}{}
%%		\eqref{s:11} \Longrightarrow \partial_t \left(\eta^{+\bld{y}} + \eta^{-\bld{y}}\right) + c_0 \partial_y \left(\eta^{+\bld{y}} - \eta^{-\bld{y}}\right) = 0\\
%%		\eqref{s:21} \Longrightarrow \partial_t \left(\eta^{+\bld{x}} + \eta^{-\bld{x}}\right) + c_0 \partial_x \left(\eta^{+\bld{x}} - \eta^{-\bld{x}}\right) =0
%%\end{subnumcases}
%%\\ \\ \\ 
%\noindent Prenons maintenant une direction donnée $\boldvec{d} = \myvector{d_1}{d_2}{}_{\mathbb{R}^2}$ avec $\left\|\boldvec{d}\right\|_2 = 1$ alors nous avons simplement
%\begin{subnumcases}{}
%	\partial_t \eta^{\bld{d_1}} + 
%	sign(d_1) c_0 \partial_x \eta^{\bld{d_1}} = -sign(d_1)\frac{c_0}{2}\partial_y \eta^{\bld{d_2}}
%\end{subnumcases}
%%
%%\begin{itemize}[label=$\mybullet$]
%%	\item Dans la direction $\pm\bld{x} : \eta^{\mp\bld{x}}, \eta^{\cdot\bld{y}} = 0$ nous avons $\eta^{\pm\bld{x}} = \eta_0^{\pm\bld{x}}\left(x \mp c_0 t\right)$
%%	\item Dans la direction $\pm\bld{y} : \eta^{\mp\bld{y}}, \eta^{\cdot\bld{x}} = 0$ nous avons $\eta^{\pm\bld{y}} = \eta_0^{\pm\bld{y}}\left(y \mp c_0 t\right)$
%%\end{itemize}
%%\vspace*{3cm}
%
%\begin{subnumcases}{}
%	\eqref{s:11} -\eqref{s:12} - \eqref{s:13} + \eqref{s:21} \Longrightarrow \partial_t \left(\eta^{+\bld{y}} + \eta^{-\bld{y}}\right) + c_0\partial_y \left(\eta^{+\bld{y}} - \eta^{-\bld{y}}\right) = 0\\
%	\eqref{s:21} -\eqref{s:22} - \eqref{s:23} + \eqref{s:11} \Longrightarrow \partial_t \left(\eta^{+\bld{x}} + \eta^{-\bld{x}}\right) + c_0\partial_x \left(\eta^{+\bld{x}} - \eta^{-\bld{x}}\right) = 0
%\end{subnumcases}
%%%%%%%%%%%%%%%%%%%%%%%%%%%%%%%%%%%%%%%%%%%%%%%%%%%%%%%%%%%%%%
%%Pour toute direction $\boldvec{d} = \myvector{d^{\bld{x}}}{d^{\bld{y}}}{}$ (avec $\left\| \boldvec{d}\right\|_2 = 1$), nous introduisons la variable caractéristique
%%\begin{equation}
%%	N^{\bld{d}} =  d^{\bld{x}} N^{\bld{x}} + d^{\bld{y}} N^{\bld{y}} = \myvector{\bld{k}}{\delta^{+}}{\delta^{-}} := \myvector{\boldvec{d} \cdot \boldvec{q} }{\frac{1}{2}\bld{k} + c_0 \zeta \left(d^{\bld{x}}+ d^{\bld{y}}\right) }{\frac{1}{2}\bld{k} - c_0 \zeta \left(d^{\bld{x}}+ d^{\bld{y}}\right) }
%%\end{equation}
%%\noindent Nous appellerons
%%\begin{equation}
%%	W^{\bld{d}} := \myvector{\zeta \left(d^{\bld{x}}+ d^{\bld{y}}\right)}{d^{\bld{x}} q_1}{d^{\bld{y}} q_2} \hspace{4mm}\text{donnant}\hspace{4mm} W^{\bld{d}}_t + A_0 \partial_x W^{\bld{d}} +  A_1 \partial_y W^{\bld{d}} = \bld{0}
%%\end{equation}
%%\noindent Nous pouvons revenir à $W$ comme
%%\begin{equation}
%%W = \boldvec{d}\cdot \myvector{P_0 N^{\bld{x}}}{P_1 N^{\bld{y}}}{} = \myvector{\frac{1}{c_0}\left[d^{\bld{x}}\left( \eta^{+\bld{x}} - \eta^{-\bld{x}}\right) + d^{\bld{y}} \left(\eta^{+\bld{y}} - \eta^{-\bld{y}}\right)\right]}{d^{\bld{x}}\left(\eta^{+\bld{x}} + \eta^{-\bld{x}}\right) + d^{\bld{y}} q_2}{d^{\bld{y}}\left(\eta^{+\bld{y}} + \eta^{-\bld{y}}\right) + d^{\bld{x}} q_1} .
%%\end{equation}
%%qui est 
%%\begin{equation}
%%	\partial_t \left(\boldvec{d} \cdot  \myvector{\eta^{+\bld{x}} - \eta^{-\bld{x}}}{\eta^{+\bld{y}} - \eta^{-\bld{y}}}{}\right) + c_0 \partial_x \left(\boldvec{d}\cdot\myvector{\eta^{+\bld{x}} + \eta^{-\bld{x}}}{\eta^{+\bld{y}} + \eta^{-\bld{y}}}{}\right)+ c_0 \partial_y \left(\boldvec{d}\cdot\myvector{\eta^{+\bld{x}} + \eta^{-\bld{x}}}{\eta^{+\bld{y}} + \eta^{-\bld{y}}}{}\right) = 0
%%\end{equation}
%%\begin{equation}
%%	\partial_t \left(\boldvec{d}\cdot \myvector{\eta^{+\bld{x}} + \eta^{-\bld{x}}}{q_2}{}\right) + c_0 \partial_x \left(\boldvec{d} \cdot  \myvector{\eta^{+\bld{x}} - \eta^{-\bld{x}}}{\eta^{+\bld{y}} - \eta^{-\bld{y}}}{}\right) = 0
%%\end{equation}
%%\begin{equation}
%%\partial_t \left(\boldvec{d}\cdot \myvector{q_1}{\eta^{+\bld{y}} + \eta^{-\bld{y}}}{}\right) + c_0 \partial_y \left(\boldvec{d} \cdot  \myvector{\eta^{+\bld{x}} - \eta^{-\bld{x}}}{\eta^{+\bld{y}} - \eta^{-\bld{y}}}{}\right) = 0
%%\end{equation}
%%SUM
%%\begin{equation}
%%\partial_t \left(\boldvec{d} \cdot  \myvector{\eta^{+\bld{x}} +q_1}{\eta^{+\bld{y}} +q_2}{}\right) 
%%+ c_0 \partial_x \left(\boldvec{d}\cdot\myvector{\eta^{+\bld{x}}}{\eta^{+\bld{y}}}{}\right)
%%+ c_0 \partial_y \left(\boldvec{d}\cdot\myvector{\eta^{+\bld{x}}}{\eta^{+\bld{y}}}{}\right) = 0
%%\end{equation}
%%\begin{equation}
%%	W = 
%%\end{equation}
%%%%%%%%%%%%%%%%%%%%%%%%%%%%%%%%%%%%%%%%%%%%%%%%%%%%%%%%%%%%%%
%Ainsi le système devient
%\begin{subnumcases}{}
%	\eta^{-}_t - c_0 \nabla \cdot \eta^{-} = 0\\
%	\eta^{+}_t + c_0 \nabla \cdot \eta^{+} = 0
%\end{subnumcases}
%Ce système d'équations découplées a pour solutions exactes
%\begin{subnumcases}{}
%	\eta^{-} (x, y, t) = \eta^{-}_0 (x+ y + c_0 t)\\
%	\eta^{+} (x, y, t) = \eta^{+}_0 (x+ y - c_0 t)
%\end{subnumcases}
%conduisant à
%\begin{subnumcases}{}
%\zeta (x, y, t) = \frac{1}{c_0}\left(\eta^{+}_0 (x+ y - c_0 t) - \eta^{-}_0 (x+ y + c_0 t)\right),\\
%\boldvec{q} (x, y, t) = \eta^{-}_0 (x+ y + c_0 t) + \eta^{+}_0 (x+ y - c_0 t).
%\end{subnumcases}
%\noindent Si nous considérons une direction $\boldvec{d} = \myvector{d_1}{d_2}{}$ alors il est toujours possible d'écrire 
%\begin{equation}
%\boldvec{d} = \myvector{d_1^{+} - d_1^{-}}{d_1^{+} - d_1^{-}}{}\hspace{3mm}\text{avec}\hspace{3mm}a^{+} = \frac{|a| + a}{2}\text{ et }a^{-} = \frac{|a| - a}{2}
%\end{equation}

%\vspace*{10cm}
%\noindent Si nous considérons une onde progressive dans une direction $\boldvec{d} = \myvector{d_1}{d_2}{}$ alors $\eta^{+}$ représente une onde progressant dans la direction $\boldvec{x}+\boldvec{y}$ tandis que $\eta^{-}$ représente une onde progressant dans la direction contraire.\\
\noindent Par ailleurs, il est toujours possible d'écrire 
\begin{equation}
\boldvec{d} = \myvector{\bld{d_1^{+}} - \bld{d_1^{-}}}{\bld{d_2^{+}} - \bld{d_2^{-}}}{}\hspace{3mm}\text{avec}\hspace{3mm}a^{+} = \frac{|a| + a}{2}\text{ et }a^{-} = \frac{|a| - a}{2}
\end{equation}
Ainsi, selon $\boldvec{d}$ nous avons\\
\begin{minipage}{0.49\textwidth}
	\begin{subnumcases}{}
	\eta^{+\bld{x}}(\boldvec{x}, t) = \frac{\bld{d_1^{+}}}{\bld{d_1}}\eta^{+\bld{x}}_0(\boldvec{x} - \boldvec{v}_{\bld{d_1}} t)\\
	\eta^{-\bld{x}}(\boldvec{x}, t) = \frac{\bld{d_1^{-}}}{\bld{d_1}}\eta^{-\bld{x}}_0(\boldvec{x} + \boldvec{v}_{\bld{d_1}} t)
	\end{subnumcases}
\end{minipage}
\begin{minipage}{0.49\textwidth}
	\begin{subnumcases}{}
	\eta^{+\bld{y}}(\boldvec{x}, t) = \frac{\bld{d_2^{+}}}{\bld{d_2}}\eta^{+\bld{y}}_0(\boldvec{x} - \boldvec{v}_{\bld{d_2}} t)\\
	\eta^{-\bld{y}}(\boldvec{x}, t) = \frac{\bld{d_2^{-}}}{\bld{d_2}}\eta^{-\bld{y}}_0(\boldvec{x} + \boldvec{v}_{\bld{d_2}} t)
	\end{subnumcases}
\end{minipage}\vspace*{3mm}\\
\noindent Ainsi selon $\boldvec{d}$ nous avons une simplification possible. Par exemple prenons $a, b\in \mathbb{R}$ et $\boldvec{d} = \left[a^2, -b^2\right]^{T}$ alors \\
\begin{minipage}{0.49\textwidth}
	\begin{subnumcases}{}
	\eta^{+\bld{x}}(\boldvec{x}, t) = \eta^{+\bld{x}}_0(\boldvec{x} - \boldvec{v}_{\bld{d_1}} t)\\
	\eta^{-\bld{x}}(\boldvec{x}, t) =0
	\end{subnumcases}
\end{minipage}
\begin{minipage}{0.49\textwidth}
	\begin{subnumcases}{}
	\eta^{+\bld{y}}(\boldvec{x}, t) =0\\
	\eta^{-\bld{y}}(\boldvec{x}, t) = \eta^{-\bld{y}}_0(\boldvec{x} + \boldvec{v}_{\bld{d_2}} t)
	\end{subnumcases}
\end{minipage}\vspace*{3mm}\\
\begin{equation}
	\myvector{\eta^{-\bld{x}}_{0}}{\eta^{+\bld{y}}_{0}}{} = \boldvec{0} \hspace{3mm}\Longrightarrow\hspace{3mm} \boldvec{q} = \boldvec{c_0}\zeta,
\end{equation}
sous la notation $\boldvec{c_0} = \myvector{c_0}{c_0}{}$, ainsi
\begin{equation}
	\boldvec{q}_t = \boldvec{c_0}\zeta_t, \hspace{4mm}\text{et}\hspace{4mm} \div[\boldvec{q}] = c_0\partial_x\zeta + c_0 \partial_y\zeta 
\end{equation}
Ainsi, le système formé par \eqref{eq:water_1}-\eqref{eq:water_3} devient
\begin{subnumcases}{}
	\zeta_t + c_0\bld{d_1} \partial_x \zeta+ c_0\bld{d_2} \partial_y \zeta= 0\\
	\zeta_t + c_0 \bld{d_1} \partial_x\zeta = 0\\
	\zeta_t + c_0 \bld{d_2} \partial_y \zeta = 0
\end{subnumcases}