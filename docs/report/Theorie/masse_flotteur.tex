\section{La masse du flotteur}
\noindent Premièrement, rappelons la seconde loi de Newton 
\begin{equation}
	\mass{b}\boldvec{a}_\mathcal{G} = \sum_{i}\,\boldvec{F_i}
\end{equation}
où $\mass{b}$ désigne la masse du flotteur et $\sum_{i}\,\boldvec{F_i}$ la somme des forces extérieures subies par le corps. Dans notre cas, seuls le poids et les forces découlant de la pression entrent en jeu, c'est-à-dire que nous avons
\begin{subnumcases}{}
\boldvec{F_1} = \mass{b}\boldvec{g}  = - g\mass{b}\boldvec{e}_z \hspace{1cm}&\text{(poids)},\\
\boldvec{F_2} = \iint_{\Omega_b}\mathcal{P}\, \normalvec\,dxdy&\text{(force issue de la pression $\mathcal{P}$)}.
\end{subnumcases}
avec $\mathcal{P} = \rho g \zeta + \bpression$ définie à \eqref{eq:pressiondefinition}. Ainsi
\begin{equation}
\mass{b}\boldvec{a}_\mathcal{G} = - g\mass{b}\boldvec{e}_z + \rho g\iint_{\Omega_b}\zeta\, \normalvec\,dxdy + \rho \iint_{\Omega_b}\bpression_{b}\, \normalvec\,dxdy \label{eq:massflotteur_1}
\end{equation}