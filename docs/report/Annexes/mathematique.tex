\chapter{Aide-mémoire}

\section{Opérateurs et coordonnées}
Dans ce rapport nous utiliserons un système de coordonnées cartésiennes $(x, y, z)$ orthonormée sur $\mathbb{R}^3$ et une variable temporelle désignée par $t$. Les systèmes de coordonnées \textit{sphériques} et \textit{cylindriques} peuvent être obtenus par simple dérivation à l'aide de la formule de dérivées en chaîne : 
\begin{equation}
	\frac{\partial y}{\partial x} = \frac{\partial y}{\partial u}\cdot\frac{\partial u}{\partial v}\cdot\frac{\partial v}{\partial x}.\label{op:chainder}
\end{equation}
Soit la quantité $\phi$ qui dépend des coordonnées spatiales et temporelle dénotée \[\phi (x, y, z, t).\]
Les équations qui gouvernent les phénomènes physiques peuvent faire apparaître des \textit{dérivées partielles de $\phi$} et en conséquence elles seront abrégées \textbf{PDEs} pour \textbf{partial differential equations}.\\

\noindent Nous utiliserons donc les notations suivantes 
\[\phi_\alpha = \partial_\alpha\phi = \frac{\partial \phi}{\partial \alpha} \lhs \text{et}\lhs \partial_\alpha^2\phi = \partial_{\alpha\alpha}\phi = \partial_\alpha\phi_\alpha = \frac{\partial^2 \phi}{\partial \alpha^2}\hspace{1cm}\alpha= x, y, z, t\]
\noindent Ainsi que de multiples opérateurs, tels que
\begin{itemize}[label=$\mybullet$]
	\item le \textbf{produit scalaire} entre $\vec{v} = \left(v_1, v_2, v_3\right)$ et $\vec{u} = \left(u_1, u_2, u_3\right)$
\[ \vec{v}\cdot\vec{u} = v_1u_1+v_2u_2+v_3u_3,\]
	\item le \textbf{produit vectoriel} entre $\vec{v}$ et $\vec{u}$
\[ \vec{v}\times\vec{u} = \myvector{v_2u_3-v_3u_2}{v_3u_1-v_1u_3}{v_1u_2-v_2u_1},\]
	\item le \textbf{produit tensoriel} entre $\vec{v}$ et $\vec{u}$
\[ \vec{v}\otimes \vec{u} = \myvector{v_1u_1\lhs v_1u_2\lhs v_1u_3}{v_2u_1\lhs v_2u_2\lhs v_2u_3}{v_3u_1\lhs v_3u_2\lhs v_3u_3}, \hspace{1cm} \vec{v}\otimes\vec{v} = \myvector{v_1^2\lhs v_1v_2\lhs v_1v_3}{v_2v_1\lhs v_2^2\lhs v_2v_3}{v_3v_1\lhs v_3v_2\lhs v_3^2},\hspace{1cm} \id = \myvector{\thvector{1}{0}{0}}{\thvector{0}{1}{0}}{\thvector{0}{0}{1}}, \]
	\item l'opérateur \textbf{gradient} pour une quantité scalaire $\phi$
\[\grad\phi = \nabla \phi = \left(\phi_x, \phi_y, \phi_z\right), \]
	\item l'opérateur de \textbf{divergence} appliqué à un vecteur $\vec{v} = \left(v_1, v_2, v_3\right)$
\[\div\vec{v} = \nabla\cdot\vec{v} = \partial_x v_1 + \partial_y v_2 + \partial_z v_3, \]
	\item l'opérateur \textbf{rotationnel} appliqué à $\vec{v}$
\[ \rot \vec{v} = \nabla \times \vec{v} = \myvector{\partial_y v_3-\partial_z v_2}{\partial_z v_1-\partial_x v_3}{\partial_x v_2-\partial_y v_1}, \]
	\item l'opérateur \textbf{laplacien}
\[\Delta\vec{v} = \partial_x^2 v_1 + \partial_y^2 v_2 + \partial_z^2 v_3. \]
	\item l'opérateur\textbf{ d'advection}
	\[\left(\vec{v}\cdot \nabla \right)\vec{v} = \myvector{v_1 \partial_x v_1 + v_2 \partial_y v_1 + v_3 \partial_z v_1}{v_1 \partial_x v_2 + v_2 \partial_y v_2 + v_3 \partial_z v_2}{v_1 \partial_x v_3 + v_2 \partial_y v_3 + v_3 \partial_z v_3} \]
\end{itemize} 
\noindent \textbf{Le théorème de Gauss (aussi appelé formule de Green), nous avons}
\begin{equation}
\iint_A \nvec \cdot \tilde{\psi}\,d A = \iiint_V \div \tilde{\psi} \,d V \label{op:greenfrm}
\end{equation}
où $\tilde{\psi}$ est un champ vectoriel et $\boldvec{n}$ un vecteur normal sortant à la surface $A$.\\
\noindent \textbf{Décomposition grâce au produit tensoriel :}
\begin{equation}
\left(\nvec\cdot \rho\velocity\right)\velocity = \nvec \cdot \rho\velocity \otimes \velocity \label{op:identitytensorvelocity}
\end{equation}
\noindent\textbf{Divergence d'un produit tensoriel :}
\begin{equation}
\div[\boldvec{v}\otimes\boldvec{u}] = \boldvec{u} \left(\div \boldvec{v}\right) + \left(\boldvec{v}\cdot \onabla{}\right)\boldvec{u} \label{op:divprodvec}
\end{equation}
\noindent\textbf{Décomposition de Lamb :}
\begin{equation}
\left(\velocity\cdot\nabla\right)\velocity = \frac{1}{2}\nabla \velocity^2 + \left(\rot \velocity\right)\times\velocity\label{op:lamb}
\end{equation}
\noindent\textbf{Théorème d'intégration de Leibniz} avec $f$, $\partial_{x, y, z}f$ continues et $\alpha (x, y, t)$ et $\beta (x, y, t)$ des fonctions différentiables de $x, y$ et $z$ et donc à $\star = (x, y)$ fixé nous avons
\begin{equation}
	\int_{\alpha(\star, t)}^{\beta(\star, t)}\, \partial_x\,f (\star, z, t)\,dz = \partial_x\left[\int_{\alpha(\star, t)}^{\beta(\star, t)}\, f (\star,z, t)\,dz\right]+ f(\star, \alpha(\star, t))\,\alpha_x (\star, t)- f(\star,\beta(\star, t), t)\,\beta_x(\star, t)\label{op:leibniz}
\end{equation}
\noindent\textbf{Changement de variable dans une intégrale} avec $f$ une fonction continue, $\Phi$ un $C^1$-difféomorphisme
\begin{equation}
	\iint_{\Phi(I)} f(\boldvec{x}) \, d\boldvec{x} = \iint_{I} f\left(\Phi(\boldvec{u})\right) \left| \text{det}\, J_{\Phi} (\boldvec{u})\right|\, d\boldvec{u}\label{op:changementdevariables}
\end{equation}
\noindent \textbf{Formule de Green dans $u, v\in\sob{1}$}
\begin{equation}
	\iiint_\Omega\,\frac{\partial u}{\partial x_k} \cdot v \, d \boldvec{x} = - \iiint_\Omega\,u\frac{\partial v}{\partial x_k}\, d \boldvec{x} + \iint_{\Gamma = \partial \Omega}\, \gamma_0\left(u\right) \cdot \gamma_0\left(v\right)\cdot \boldvec{n}_k\,d \boldvec{\sigma},\hspace{1cm}k=1,\dots, N. \label{op:greenh1}
\end{equation}
\noindent \textbf{Formules de Green dans $\sob{2}$}\\
\begin{subequations}
	$\mybullet$ $u, v\in\sob{2}$ 
	\begin{equation}
	\iiint_\Omega\,\Delta u \cdot v \, d \boldvec{x} =  \iiint_\Omega\,u\cdot\Delta v\, d \boldvec{x} + \iint_{\Gamma = \partial \Omega}\, \left[\frac{\partial u }{\partial \boldvec{n}} \cdot \frac{\partial v}{\partial \boldvec{n}}\right]\,d \boldvec{\sigma}. \label{op:greenh2_delta}
	\end{equation}
	$\mybullet$ $u\in\sob{2}$ et $v\in\sob{1}$ 
	\begin{equation}
	\iiint_\Omega\,\Delta u \cdot v \, d \boldvec{x} = -  \iiint_\Omega\,\nabla u\cdot\nabla v\, d \boldvec{x} + \iint_{\Gamma = \partial \Omega}\, \left[\frac{\partial u }{\partial \boldvec{n}} \cdot \gamma_0\left(v\right)\right]\,d \boldvec{\sigma}. \label{op:greenh2_nablas}
	\end{equation}
\end{subequations}
\section{Quantités physique}
Pour un fluide nous pouvons être amené à étudier différentes quantités physiques telles que
\begin{itemize}[label=$\mybullet$]
	\item sa masse notée $m$,
	\item la masse volumique $\rho$,
	\item sa vitesse $\velocity = \left(u, v, w\right)$,
	\item la quantité de mouvement $\vec{p} = \rho\velocity$,
	\item la pression thermodynamique $P$,
	\item l'énergie total par unité de masse, notée $E$,
	\item l'énergie interne massique $e$, 
	\item l'entropie $s$,
	\item l'enthalpie $\boldsymbol{h}$,
	\item la température $T$, et
	\item l'accélération gravitationnelle terrestre $g$ de $9.81\,m/s^2$.
\end{itemize}
La pression $P$, le volume et la température $T$ forment les \textbf{variables d'état} d'un système. Chacune de ces variables peut être extensive ou intensive, respectivement définie sur l'ensemble du système considéré ou considérée égale en tout point du système.\\
Les autres variables apparaissant dans le système peuvent toujours être exprimée à partir des variables d'état par une \textbf{fonction d'état}.