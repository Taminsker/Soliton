\section{Quadratures}

\subsection{Quadrature sur un élément de référence de type ligne}

\begin{table}[H]
	\centering
	\footnotesize
	\begin{tabular}{|>{\bfseries}c|c|c|c|}
		\toprule
		\textbf{n} & \textbf{Points d'intégration} & \textbf{Poids d'intégration} & \textbf{Degré de}\\
		& $\xi_i$ & $w_i$ & \textbf{précision}\\
		\midrule
		1 & 0 &  2 & 1\\
		\midrule
		2 & - 0,577 350 269 189 625 & 1 & 3\\
		 & + 0,577 350 269 189 625 & & \\
		 \midrule
		 3 & - 0,774 596 669 241 483 & 0,555 555 555 555 556 & 5\\
		 & 0,0 & 0,888 888 888 888 889 & \\
		 & + 0,774 596 669 241 483 & 0,555 555 555 555 556 & \\
		 \midrule
		 4 & - 0,861 136 311 594 052 & 0,347 854 845 137 454 & 7\\
		   & - 0,339 981 043 584 856 & 0,652 145 154 862 545 & \\
		   & + 0,339 981 043 584 856 & 0,652 145 154 862 545 & \\
		   & + 0,861 136 311 594 052 & 0,347 854 845 137 454 & \\
		\midrule		
		5 & - 0,906 179 845 938 664 & 0,236 926 885 056 189  & 9\\
		& - 0,538 469 310 105 683  & 0,478 628 670 499 365 & \\
		& 0,0 & 0,568 888 889 888 889 & \\
		& + 0,538 469 310 105 683  & 0,478 628 670 499 365 & \\
		& + 0,906 179 845 938 664 & 0,236 926 885 056 189  & \\ 
		\bottomrule
		\end{tabular}
		\label{tab:quad1d}
		\caption{Quadrature de Gauss-Legendre (1D) (source \citet{fortin_les_2020})}
\end{table}

\subsection{Quadrature sur un élément de référence de type triangle}
\begin{table}[H]
	\centering
	\footnotesize
	\begin{tabular}{|>{\bfseries}c|c|c|c|c|}
		\toprule
		\textbf{n} & \multicolumn{2}{c|}{\textbf{Points d'intégration}} & \textbf{Poids d'intégration} & \textbf{Degré de}\\
		& $\xi_i$ & $\eta_i$ & $w_i$ & \textbf{précision}\\
		\midrule
		1 & +0,333 333 333 333 3333 & +0,333 333 333 333 3333 & +0,500 000 000 000 0000 & 1\\
		\midrule
		3 & +0,666 666 666 666 6667 & +0,166 666 666 666 6667 & +0,166 666 666 666 6667 & 2\\
		 & +0,166 666 666 666 6667 & +0,666 666 666 666 6667 & +0,166 666 666 666 6667 & \\
		 & +0,166 666 666 666 6667 & +0,166 666 666 666 6667 & +0,166 666 666 666 6667 & \\
		\midrule
		4 & +0,333 333 333 333 3333 & +0,333 333 333 333 3333 & -0,281 250 000 000 0000 & 3\\
		  & +0,2 & +0,2 & +0,260 416 666 666 6667 & \\
          & +0,2 & +0,6 & +0,260 416 666 666 6667 & \\
          & +0,6 & +0,2 & +0,260 416 666 666 6667 & \\
        \midrule
        6 & +0,108 103 018 168 070 & +0,445 948 490 915 965 & +0,111 690 794 839 0055 & 4\\
          & +0,445 948 490 915 965 & +0,108 103 018 168 070 & +0,111 690 794 839 0055 & \\
          & +0,445 948 490 915 965 & +0,445 948 490 915 965 & +0,111 690 794 839 0055 & \\
          & +0,816 847 572 980 459 & +0,091 576 213 509 771 & +0,054 975 871 827 6610 & \\
          & +0,091 576 213 509 771 & +0,816 847 572 980 459 & +0,054 975 871 827 6610 & \\
          & +0,091 576 213 509 771 & +0,091 576 213 509 771 & +0,054 975 871 827 6610 & \\
        \midrule
        12 & +0,873 821 971 016 996 & +0,063 089 014 491 502 & +0,025 422 453 185 1035 & 6\\
           & +0,063 089 014 491 502 & +0,873 821 971 016 996 & +0,025 422 453 185 1035 & \\
           & +0,063 089 014 491 502 & +0,063 089 014 491 502 & +0,025 422 453 185 1035 & \\
           & +0,501 426 509 658 179 & +0,249 286 745 170 910 & +0,058 393 137 863 1895 & \\
           & +0,249 286 745 170 910 & +0,501 426 509 658 179 & +0,058 393 137 863 1895 & \\
           & +0,249 286 745 170 910 & +0,249 286 745 170 910 & +0,058 393 137 863 1895 & \\
           & +0,636 502 499 121 399 & +0,310 352 451 033 785 & +0,041 425 537 809 1870 & \\
           & +0,636 502 499 121 399 & +0,053 145 049 844 816 & +0,041 425 537 809 1870 & \\
           & +0,310 352 451 033 785 & +0,636 502 499 121 399 & +0,041 425 537 809 1870 & \\
           & +0,310 352 451 033 785 & +0,053 145 049 844 816 & +0,041 425 537 809 1870 & \\
           & +0,053 145 049 844 816 & +0,310 352 451 033 785 & +0,041 425 537 809 1870 & \\
           & +0,053 145 049 844 816 & +0,636 502 499 121 399 & +0,041 425 537 809 1870 & \\
		\bottomrule
	\end{tabular}
	\label{tab:quad2d}
	\caption{Quadrature de Hammer (2D) (source \citet{fortin_les_2020})}
\end{table}